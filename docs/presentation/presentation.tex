\documentclass{beamer}
\usepackage{graphicx}

\mode<presentation>
{
  \usetheme{Frankfurt}
  \setbeamercovered{transparent}
}

\setcounter{tocdepth}{1}

\title[Water Compression]{Improved Data Compression of Molecular Dynamics of
  Water}

%\subtitle{}

\author{Keegan Smith
  \and Julian Kenwood
  \and Min-Young Wu}

\titlegraphic{\includegraphics[height=16mm]{cslogo}}

\date{24 November 2009}

\institute[UCT]{Department of Computer Science \\ University of Cape Town}

\AtBeginSection[]
{
  \begin{frame}
    \frametitle{Outline}
    \tableofcontents[currentsection]
  \end{frame}
}

\begin{document}

\begin{frame}
  \titlepage
\end{frame}


\begin{frame}{Overview}
  \tableofcontents
\end{frame}


% {{{
\section{Introduction}
\begin{frame}{Introduction}
  \begin{itemize}
  \item Molecular Dynamics Simulations are simulations representing the
    interactions of atoms over a period of time.

  \item Many such simulations contain a large amount of water molecules, often
    between 50\% to 90\% of the atoms are part of a water molecule.

  \item These simulations can represent a very large time period, so they can
    be very large: From $100$MB to $1$TB.

  \item The aim of our project was to develop compression schemes that are
    able to take advantage of this large amount of water.
  \end{itemize}
\end{frame}


\begin{frame}{Introduction}
  \begin{figure}
    \includegraphics[width=0.5\textwidth]{julian-images/rabies.png}
    \caption{A Rabies Virus in water. There is a total of $450\,000$ atoms,
      68\% of which is water.}
  \end{figure}
\end{frame}
% }}}

\begin{frame}{Work Breakdown}
  \begin{figure}
    \includegraphics[width=1.0\textwidth]{legends.png}
  \end{figure}
\end{frame}


\section{Visualisations and Quantisation}

\begin{frame}{Visualisations}
% {{{
\begin{enumerate}

  \item Water Point
  \begin{itemize}
    \item Water molecules represented by points.
  \end{itemize}

  \item Ball-and-stick
  \begin{itemize}
    \item Atoms represented by spheres.
  \end{itemize}

  \item Metaballs
  \begin{itemize}
    \item Boundary between water and non-water regions extracted.
  \end{itemize}

  \item Water Cluster
  \begin{itemize}
    \item Water molecules are clustered together.
  \end{itemize}

  \item Quantisation Error
  \begin{itemize}
    \item Shows introduced quantisation error.
  \end{itemize}

\end{enumerate}
\end{frame}
% }}}

\begin{frame}{Visualisation}{Water Point}
% {{{
\begin{figure}
  \centering
  \includegraphics[width=55mm]{min-images/water-point.png}
  \caption{Each water molecule is represented by a single point.}
\end{figure}
\end{frame}
% }}}

\begin{frame}{Visualisation}{Ball-and-stick}
% {{{
\begin{figure}
  \centering
  \includegraphics[width=55mm]{min-images/ball-and-stick.png}
  \caption{Molecules are represented by colour coded spheres, with the cylinders representing the bonds between atoms.}
\end{figure}
\end{frame}
% }}}

\begin{frame}{Visualisation}{Metaballs}
% {{{
\begin{figure}
  \centering
  \includegraphics[width=55mm]{min-images/metaballs.png}
  \caption{The boundary between the water and non-water regions of the volume are extracted and rendered.}
\end{figure}
\end{frame}
% }}}

\begin{frame}{Visualisation}{Water Cluster}
% {{{
\begin{figure}
  \centering
  \includegraphics[width=55mm]{min-images/water-cluster.png}
  \caption{Water molecules within a cluster are joined together using cylinders.}
\end{figure}
\end{frame}
% }}}

\begin{frame}{Visualisation}{Quantisation Error}
% {{{
\begin{figure}
  \centering
  \includegraphics[width=55mm]{min-images/quantisation-error-line.png}
  \caption{The error introduced from quantisation is colour coded on a gradient.}
\end{figure}
\end{frame}
% }}}

\begin{frame}{Quantisation}
% {{{
To aid compression, the floating point data is first converted to integer values.

\begin{figure}
  \centering
  \includegraphics[width=75mm]{min-images/quantisation.png}
  \caption{Quantising point data can be seen as snapping points to a grid.}
\end{figure}
\end{frame}
% }}}

\begin{frame}{Quantisation Experiment}
% {{{

\begin{itemize}

  \item Quantisation causes a loss in data.

  \item A quantisation experiment was conducted to measure the perceptually visible effects of quantisation.

  \item The experiment helps determine an appropriate level of quantisation, tradeoff between compression ratio and discarded data.

  \item What is the level of quantisation where the perceived differences between the original and quantised data are not significantly different.

\end{itemize}

\end{frame}
% }}}

\begin{frame}{Quantisation Effects}
% {{{
\begin{figure}
  \centering
  \includegraphics[width=70mm]{min-images/ball-and-stick-4680.png}
  \caption{Quantisation effects for the ball-and-stick visualisation.}
\end{figure}
\end{frame}
% }}}

\begin{frame}{Quantisation Effects}
% {{{
\begin{figure}
  \centering
  \includegraphics[width=70mm]{min-images/metaballs-4680.png}
  \caption{Quantisation effects for the metaballs visualisation.}
\end{figure}
\end{frame}
% }}}

\begin{frame}{Quantisation Results}
% {{{
\begin{figure}
  \centering
  \includegraphics[width=100mm]{min-images/bm-means.png}
  \caption{Graph showing the mean ratings for each of the quantisation levels and visualisation techniques.}
\end{figure}
\end{frame}
% }}}

\begin{frame}{Quantisation Results}
% {{{
\begin{itemize}

  \item 4 bit quantisation was rated significantly different.

  \item 6 bit quantisation was rated moderately different.

  \item 8 and 10 bit quantisation were rated not significantly different.

\end{itemize}
\end{frame}
% }}}


% {{{
\section{Intraframe}
\begin{frame}{What We Implemented}
  \begin{itemize}
    \item Reference Point Cloud Compressors
      \begin{itemize}
        \item Devillers and Gandoin
        \item Gumhold et.~al.
      \end{itemize}
    \item Reference General MD Compressor by Omeltchenko et.~al.~(Implemented
      by Julian)
    \item Reference compressor using quantisation and gzip
    \item Our own scheme is a predictive point cloud compressor which targets
      water.
  \end{itemize}
\end{frame}

\begin{frame}{Water Compression Scheme}{Predictors}
  \begin{figure}[h]
    \centering \includegraphics[width=0.6\textwidth]{keegan-images/h402}
  \end{figure}
  \begin{figure}[h]
    \centering \includegraphics[trim = 10mm 30mm 10mm 45mm, clip,
      width=0.6\textwidth]{keegan-images/dimer-angle}
  \end{figure}
\end{frame}

\begin{frame}{Water Compression Scheme}{High-level Overview}
  \begin{enumerate}
  \item Separate the water and non-water atoms and compress separately.
  \item Create graph of water molecules.
  \item Create a spanning tree from the graph.
  \item Encode the tree, predictor used and residuals. This is done in BFS
    order.
  \end{enumerate}
\end{frame}

\begin{frame}{Permutation Encoding}{Problem}
  \begin{itemize}
  \item Many of the schemes (including our scheme) do not preserve the order
    of the points in the file.
  \item For example in Devillers and Gandoin
    \[ [ (1, 2), (1, 3), (0, 0), (2, 3), (2, 2), (0, 2), (1, 1) ] \]
    when decompressed becomes
    \[ [ (0, 0), (1, 1), (0, 2), (1, 2), (1, 3), (2, 2), (2, 3) ] \]
  \item The file format requires the original order of the points to be
    recovered.
  \item Encoding the order of the points became a significant portion of the
    compressed file.
  \end{itemize}
\end{frame}

\begin{frame}{Permutation Encoding}{Solution}
  \begin{itemize}
  \item We explored five different techniques.
  \item We derived the lower bound for the worst case performance of a general
    permutation encoder.
    \begin{equation*}
      \displaystyle\sum^{N}_{i=1} \log_2(i)
    \end{equation*}
  \item We showed that our scheme, \textbf{optimal}, encodes at roughly the
    lower bound.
  \item Empirically our scheme, \textbf{delta}, performed best since its worse
    case is unlikely to occur.
  \end{itemize}
\end{frame}

\begin{frame}{Results}
  \begin{figure}[h]
    \centering \includegraphics[trim = 10mm 30mm 10mm 35mm, clip,
      width=0.9\textwidth]{keegan-images/results-average}
  \end{figure}
  \begin{itemize}
  \item In some test cases (with 100\% water), our scheme had a 11\% gain on
    the next best intraframe scheme.
  \end{itemize}
\end{frame}


\section{Interframe}

\begin{frame}{Overview}
\begin{itemize}
 \item Interframe Compression is a technique where atoms are compressed using
   knowledge about their previous positions.

 \item Five Interframe Compression schemes were implemented and tested:
 \begin{itemize}
   \item Polynomial Extrapolation
   \item Spline Extrapolation
   \item Smallest Error Encoding
   \item Common Error Encoding
   \item k-Nearest Neighbour Encoding
 \end{itemize}
\end{itemize}
\end{frame}

\begin{frame}{Polynomial Extrapolation}
  \begin{figure}[h]
    \centering \includegraphics[width=0.6\textwidth]{julian-images/p1.png}
  \end{figure}
\begin{itemize}
 \item Uses the Lagrange Interpolation formula to construct a polynomial that
   passes through all of the previous atom positions.

 \item Using this polynomial, a prediction is made as to where the atom is
   likely to be.
\end{itemize}
\end{frame}

\begin{frame}{Spline Extrapolation}
  \begin{figure}[h]
    \centering \includegraphics[width=0.6\textwidth]{julian-images/p2.png}
  \end{figure}
\begin{itemize}
 \item A similar scheme to Polynomial Extrapolation.

 \item Uses a Bezi\'er curve to construct a polynomial.
\end{itemize}
\end{frame}

\begin{frame}{Smallest Error Encoding}
\begin{figure}[h]
    \centering \includegraphics[width=0.6\textwidth]{julian-images/p3.png}
  \end{figure}
\begin{itemize}

 \item Treats the memory as a databank of past atom positions.

 \item The atom is predicted at its closest previous position.
\end{itemize}
\end{frame}

\begin{frame}{Common Error and kNN Encoding}

\begin{itemize}
 \item Common Error Encoding
\begin{itemize}
 \item An enhancement of Smallest Error Encoding.
 \item Heuristic functions are used for better compression.
\end{itemize}

\item k-Nearest Neighbour Encoding
\begin{itemize}
 \item Uses a learning scheme to attempt to find patterns in the atoms' positions.
\end{itemize}
\end{itemize}
\end{frame}

\begin{frame}{Testing}
\begin{itemize}
 \item Several different molecular simulations were used. Specifically tested:
 \begin{itemize}
   \item Size of simulation: Number of atoms and frames.
   \item Coherency of motion: Temperature and simulation timestep.
   \item Simulation makeup: Water percentage.
 \end{itemize}
 \end{itemize}
\end{frame}

\begin{frame}{Results}
\begin{figure}[h]
\centering \includegraphics[trim = 10mm 60mm 10mm 5mm, clip,
  width=0.9\textwidth]{julian-images/interframe_results.pdf}
\end{figure}
\begin{itemize}
\item Our best scheme beats the current best scheme by approximately 20\%.
\end{itemize}
\end{frame}


\section{Conclusion}
\begin{frame}{Conclusion}{Research Questions}

\textbf{Visualisation and quantisation}

\ \newline
Acceptable level of quantisation?
\begin{itemize}
  \item 8 bit quantisation recommended for visual data.
\end{itemize}

\ \newline
Effectively visualise and cope with large levels of detail?
\begin{itemize}
  \item Yes. Filtering applied to reduce clutter.
\end{itemize}
\end{frame}

\begin{frame}{Conclusion}{Research Questions}
\textbf{Intraframe}

\ \newline
Can we effectively exploit what we know about water?
\begin{itemize}
  \item Yes. The more water we have, the better our compression.
\end{itemize}

\ \newline
Can we improve on other methods?
\begin{itemize}
  \item Yes. We improve on all the methods tested.
\end{itemize}

\ \newline
Can we Compress and Decompress in an efficient manner?
\begin{itemize}
  \item Yes. $O(n \log n)$ and $O(n)$ per frame respectively.
\end{itemize}
\end{frame}

\begin{frame}{Conclusion}{Research Questions}
\textbf{Interframe}

\ \newline
Can interframe prediction be used for compression?
\begin{itemize}
  \item Yes.
\end{itemize}

\ \newline
Achieve better compression rates than other methods?
\begin{itemize}
  \item Yes, specifically in the case of water-dominant simulations.
\end{itemize}
\end{frame}

\begin{frame}{Conclusion}
\begin{itemize}
 \item 8 bit quantisation recommended for visual fidelity.
 \item Exploiting the structure of water is a viable technique for compression.
 \item Exploiting the correlation between atom positions between frames is a viable technique for compression.
\end{itemize}
\end{frame}

\begin{frame}{Future Work}
  Enhancing
  \begin{itemize}
  \item Better joining of components in spanning tree for intraframe water
    compressor.
  \item More efficient implementations of interframe compressors.
  \end{itemize}
  Extending
  \begin{itemize}
  \item Parallel intraframe compressor.
  \item General molecular dynamics compressor.
  \item Point cloud compressor which preserves point order.
  \item Test more interframe compression schemes.
  \end{itemize}
\end{frame}

\begin{frame}{Thank you}
  \begin{center}
    Any Questions?
  \end{center}
\end{frame}
% }}}

\end{document}

% LocalWords:  intraframe
