\thispagestyle{empty}

\section*{Abstract}
% {{{

Molecular simulations generates large amounts of data, over many days. Being
able to compress this data will be advantageous; both for storage, as well as
data transmission. Performing compression by taking advantage of certain
characteristics of the data is expected to yield promising results. The water
compression project aims to achieve high compression ratios by exploiting the
presence of large amounts of water in the simulation.

The visualisation aspect of the water compression project is there to support
the main compression goal of the system. To facilitate compression, the
molecular data is quantised, floating point values gets mapped to integer
values. Thus, the visualisation aspect has been split up into two main parts:
visualisation of the molecular data, and an experiment to measure the
perceptual impact of quantisation.

A number of different visualisation techniques has been developed in order to
visualise the molecular data, two of which was used in the quantisation
experiment. The quantisation experiment results indicates that 8 and 10 bit
quantisation yields results that are perceived to be similar to the original
data. Further quantisation, 6 and 4 bit quantisation, yields results that are
rated different and very different.

\ \\
\ \\
\ \\

\textbf{Categories:} \\
H.5.2 [User Interfaces] Graphical user interfaces (GUI)

\textbf{Keywords:} \\
Visualisation

% }}}

