\thispagestyle{empty}

\begin{abstract}
% {{{

Molecular simulations generate large amounts of data, over many days. The
ability to compress this data will be advantageous; both for storage, as well
as data transmission. As molecular simulations use water as the solvent for the
simulation, there are large amounts of water present in the volume. The water
compression project thus aims to achieve high compression ratios by exploiting
certain structural properties of water to effectively compress water dominant
molecular simulations.

The visualisation aspect of the water compression project is there to support
the main compression goal of the system, the compression of molecular
simulations. To facilitate compression, the molecular data is quantised:
floating point values are mapped to integer values. The visualisation component
of the system has thus been split up into two main parts: visualisation of the
molecular data, and an experiment to measure the perceptual impact of
quantisation. The quantisation experiment is used to recommend a quantisation
level for data used for viewing.

A number of different visualisation techniques have been developed in order to
visualise the molecular data, two of which were used in the quantisation
experiment. The quantisation experiment tested 4 different quantisation levels:
4, 6, 8 and 10 bit quantisation levels. The results indicate that 8 and 10 bit
quantisation yields results that are perceived to be similar to the original
data. Further quantisation to 6 and 4 bits, yields results that are rated
significantly different from the original data. Thus, 8 bit quantisation is the
recommended quantisation level to use for data used for visual inspection.

% }}}
\end{abstract}

