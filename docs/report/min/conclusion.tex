\chapter{Conclusion}
% {{{
\label{cha:conclusion}

The visualisation aspect of the water compression system is split into two main
parts. The first part involves visualising the molecular data; both for general
viewing, as well as for supporting the compression aspect of the system. The
second part tests the perceptual impact of quantisation, and allows a decision
on an appropriate level of quantisation to be made.

\section{Visualisation Techniques}
% {{{
\label{sec:conclusion_visualisation}

Five different visualisation techniques have been successfully implemented:
\begin{itemize}

  \item Water Point Visualisation - allows for large areas of water and
  non-water to be see in overview and will be useful for getting an initial
  idea of the volume.

  \item Ball-and-Stick Visualisation - is the traditional approach to
  visualising molecular data, the atoms are represented by spheres and the
  bonds are represented by cylinders.

  \item Metaballs Visualisation - extracts the surface between the water and
  non-water regions of the volume. These regions are clearly separated from
  each other.

  \item Water Cluster Visualisation - shows the extracted water clusters in the
  volume, and is used by the intraframe compression component of the system.

  \item Quantisation Error Visualisation - shows the errors introduced by
  quantisation, which are evenly distributed across the volume.

\end{itemize}

% }}}

\section{Quantisation Experiment}
% {{{
\label{sec:conclustion_experiment}

The results from the quantisation experiment showed that 10 bit and 8 bit
quantisation yield very similar results, and are not rated significantly
different from the original data. Quantisation using 6 bits is still similar,
but the differences are more noticeable. While 4 bit quantisation is rated as
significantly different from the original data, thus, this is thus not
recommended for general use.

The ball-and-stick and metaballs visualisation techniques perform very
similarly. The ball-and-stick visualisation yields more noticeable differences
when using more bits compared to metaballs, but the metaballs yields more
differences when using fewer bits.

We recommend a maximum quantisation level of 8 bits for visual fidelity. For
large simulations where higher fidelity is needed, then more bits may be used,
but for general viewing, 8 bit quantisation is sufficient.

% }}}

\section{Future work}
% {{{
\label{sec:conclusion_future}

Only two visualisation techniques were tested in the quantisation experiment,
different visualisations may be affected by quantisation differently. Thus,
further experimentation could be conducted to evaluate the effects of
quantisation on other molecular visualisations.

Researchers using the molecular simulations may want to do analysis on the
data, a different quantisation level may be needed due to the different
requirements. Experts in the field would need to be consulted to determine an
acceptable level of quantisation error that can be introduced, this would thus
allow an appropriate quantisation level to be recommended for data to be used
in analysis.

Only one of the participants that took part in the quantisation experiment has
a significant background in chemistry. If the participants has more knowledge
of chemistry, they may be able to better notice the effects of quantisation. An
experiment could be conducted using chemists as participants to determine if
their responses are significantly different. The results from this experiment
may change the recommended level of quantisation.

% }}}

% }}}

