\chapter{Conclusion}
% {{{
\label{cha:conclusion}

The visualisation aspect of the water compression system is split into two main
parts. The first part is visualising the molecular data; both for general
viewing, as well as for supporting the compression aspect of the system. The
second part is testing the perceptual impact of quantisation, and hence decide
on an appropriate level of quantisation.

\section{Visualisation Techniques}
% {{{
\label{sec:conclusion_visualisation}

5 different visualisation techniques has been successfully implemented:
\begin{itemize}
  \item Water Point Visualisation
  \item Ball and Stick Visualisation
  \item Metaballs Visualisation
  \item Water Cluster Visualisation
  \item Quantisation Error Visualisation
\end{itemize}

The Water Point Visualisation allows for large areas of water and non-water to
be seen quickly, this will be useful for getting an initial idea of the volume.

The Ball and Stick Visualisation renders each of the atoms, thus providing the
most detail.

The Metaballs Visualisation determines and extracts the surface between the
water and non-water regions of the volume. The water and non-water regions are
clearly separated from each other.

The Water Cluster Visualisation shows the extracted water clusters in the
volume, which is used by the compression aspect of the system.

The Quantisation Error Visualisation shows the errors introduced by
quantisation, which are evenly distributed across the volume.

% }}}

\section{Quantisation Experiment}
% {{{
\label{sec:conclustion_experiment}

The results from the quantisation experiment showed that 10 bit and 8 bit
quantisation yield very similar results, and is not rated considerably
different from the original data.  6 bit quantisation is still similar, but the
effects are more noticeable. While 4 bit quantisation is rated very different
from the original data and is thus not recommended for general use.

The recommended quantisation level to use is 8 bits.  For large simulations
where higher fidelity is needed, then more bits may be used, but for general
viewing, 8 bit quantisation is sufficient.

The results between the ball and stick, and metaballs visualisation technique
is very similar.

% }}}

\section{Future work}
% {{{
\label{sec:conclusion_future}

% }}}

% }}}

