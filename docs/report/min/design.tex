\chapter{Design}
% {{{
\label{chap:Design}

The design of the system has been split up into a number of components; this is
to help manage the development of the system, both time wise and resources
wise.

The main component categories of the system are:
\begin{itemize}
  \item Simulation I/O
  \item Coders
  \item Compressors
  \item Visualisation
  \item General components
  \item Method specific components
  \item Verification components
\end{itemize}

Each of the component category will be briefly discussed in the following sections.

% TODO: add image of components, and how they link up

\section{Simulation I/O}
% {{{
\label{sec:simulationio}

This category is to handle the input and output of molecular simulation data.
The most notable uses will be the PDB and DCD file loaders.

\paragraph{PDB Reader}
% {{{

The PDB file contains the molecular information for the simulation, i.e. what
are the atoms and other identifying information. The main use of the PDB file
will be finding where the water molecules are.

% }}}

\paragraph{DCD Reader}
% {{{

The DCD file contains the actual simulation data, i.e. where the atoms are at
each of the frames. This component must be able to seek to arbitrary frames
within the file for visualisation purposes.

% }}}

% }}}

\section{Coders}
% {{{
\label{sec:coders}

The coder components will be used to encode/decode the symbols to and from
their compressed bit representations.

\paragraph{Arithmetic coder}
% {{{

The arithmetic coder component will be an adaptive arithmetic coder and will be
able to handle large numbers of symbols.

% }}}

\paragraph{Omeltchenko coder}
% {{{

The Omeltchenko coder will be a simple adaptive encoder that will be able to
encode integers. All data to be encoded will first be converted to an integer
format.

% }}}

% }}}

\section{Compressors}
% {{{
\label{sec:compressors}

The compressor component will be the driver behind the compression and
decompression of the files. It will be responsible for calling and
co-ordinating the compression stages, maintaining the necessary state
information, as well as the file operations.

% }}}

\section{Visualisation}
% {{{
\label{sec:visualistioncomponent}

The visualiation component will be used to visualise the water molecules being
compressed and decompressed. The visualisation component will be a purely
readonly operation on the data, it will not be producing any data.

% }}}

\section{General components}
% {{{
\label{sec:generalcomponents}

These components will be used to perform general transformations needed for the
different encoders.

\paragraph{Quantiser}
% {{{

The quantiser component will be used to quantise the positions of the atoms to
a grid of predetermined granularity. The atom positions will be snapped to the
nearest grid cells as necessary. The quantiser will convert from floating point
values to integer points values, which will be used by the various encoders.

% }}}

\paragraph{Frame splitter}
% {{{

As the system will be mostly interested in the water molecules of the molecular
simulation, the frame splitter component will be used to separate out the water
molecules from the simulation data. The output will be used by both the
visualistion and compression components.

% }}}

\paragraph{Interframe prediction coder}
% {{{

The interframe prediction coder will be used to reduce the running time of the
compressors by using temporal coherence of the data, the position of atoms will
be unlikely to travel far from their previous position. This can be taken
advantage of by using a simple estimate of the velocity of the atom, and then
using that to predict the position of the atom in the next frame.

% }}}

% }}}

\section{Method specific components}
% {{{
\label{sec:methodcomponents}

The various compression schemes will have their own specific components, this
category contains those specific components.

\paragraph{Omeltchenko}
% {{{

The Omeltchenko compression schemes uses an octree indexer to sort and
partition the atom positions. A delta encoder/decoder is then used to encode
and decode the stream of octree indices.

% }}}

\paragraph{Gandoin \& Devilliers}
% {{{

A point to list encoder is used by the Gandoin \& Devilliers compression
scheme. The point data will be split using KD-tree subdivisions, which will
then be encoded.

%TODO: fill out more

% }}}

\paragraph{Predictive point cloud}
% {{{

The predictive point cloud components includes the following components: a
graph creator, a spanning tree creator and a tree serialiser. The graph creator
produces a graph of water molecules, which is then pruned by the spanning tree
component. The spanning tree is finally serialised with the tree serialiser,
and the output of that is compressed using arithmetic encoding.

% }}}

% }}}

\section{Verification components}
% {{{
\label{sec:verificationcomponents}

The verification components will be used to verify and record how the various
aspects of the system are performing.

\paragraph{Transform verifier}
% {{{

The transform verifier is a simple component to test if the compression is
lossless or not.

% }}}

\paragraph{Quantiser statistics}
% {{{

This component will collect various statistics from the quantising step of the
compression.

% }}}

\paragraph{Prediction statistics}
% {{{

This component will collect various statistics from the prediction step of the
compression. These statistics will be used to evaluate how effective the
heuristics used for prediction are.

% }}}

% }}}

% }}}

