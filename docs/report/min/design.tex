\chapter{Design}
% {{{
\label{chap:Design}

The design of the system has been split up into a number of components; this is
to help manage the development of the system, both time wise and resources
wise.

The main component categories of the system are:
\begin{itemize}
  \item Simulation I/O
  \item Coders
  \item Compressors
  \item Visualisation
  \item General components
  \item Method specific components
  \item Verification components
\end{itemize}

Each of the component category will be briefly discussed in the following sections.

\section{Simulation I/O}
% {{{
\label{sec:simulationio}

This category is to handle the input and output of molecular simulation data.
The most notable uses will be the PDB and DCD file loaders.

\paragraph{PDB Reader}
% {{{

The PDB file contains the molecular information for the simulation, i.e. what
are the atoms and other identifying information. The main use of the PDB file
will be finding where the water molecules are.

% }}}

\paragraph{DCD Reader}
% {{{

The DCD file contains the actual simulation data, i.e. where the atoms are at
each of the frames. This component must be able to seek to arbitrary frames
within the file for visualisation purposes.

% }}}

% }}}

\section{Coders}
% {{{
\label{sec:coders}

The coder components will be used to encode/decode the symbols to and from
their compressed bit representations.

\paragraph{Arithmetic coder}
% {{{

The arithmetic coder component will be an adaptive arithmetic coder and will be
able to handle large numbers of symbols.

% }}}

\paragraph{Omeltchenko coder}
% {{{

The Omeltchenko coder will be a simple adaptive encoder that will be able to
encode integers. All data to be encoded will first be converted to an integer
format.

% }}}

% }}}

\section{Compressors}
% {{{
\label{sec:compressors}

The compressor component will be the driver behind the compression and
decompression of the files. It will be responsible for calling and
co-ordinating the compression stages, maintaining the necessary state
information, as well as the file operations.

% }}}

\section{Visualisation}
% {{{
\label{sec:visualistioncomponent}

The visualiation component will be used to visualise the water molecules being
compressed and decompressed. The visualisation component will be a purely
readonly operation on the data, it will not be producing any data.

% }}}

\section{General components}
% {{{
\label{sec:generalcomponents}

These components will be used to perform general transformations needed for the
different encoders.

\paragraph{Quantiser}
% {{{

The quantiser component will be used to quantise the positions of the atoms to
a grid of predetermined granularity. The atom positions will be snapped to the
nearest grid cells as necessary. The quantiser will convert from floating point
values to integer points values, which will be used by the various encoders.

% }}}

\paragraph{Frame splitter}
% {{{

As the system will be mostly interested in the water molecules of the molecular
simulation, the frame splitter component will be used to separate out the water
molecules from the simulation data. The output will be used by both the
visualistion and compression components.

% }}}

\paragraph{Interframe prediction coder}
% {{{

The interframe prediction coder will be used to reduce the running time of the
compressors by using temporal coherence of the data, the position of atoms will
be unlikely to travel far from their previous position. This can be taken
advantage of by using a simple estimate of the velocity of the atom, and then
using that to predict the position of the atom in the next frame.

% }}}

% }}}

\section{Method specific components}
% {{{
\label{sec:methodcomponents}

\paragraph{Octree indexer}
% {{{

This component will calculate the octree index of each of the atoms in the
frame, the Omelchenko coder will then be used to compress this output.

% }}}

\paragraph{Delta encoder/decoder}
% {{{

The delta encoder is the last stage of the Omeltchenko compression scheme. This
component will compress a stream of octree indices, which will come from the
octree indexer component.

% }}}

\paragraph{Point to list encoder}
% {{{

The Gandoin \& Devilliers compression scheme will use this component to
recursively split the molecular data into KD-tree subdivisions.

%TODO: fill out more

% }}}

% TODO: combine the graph creator, spanning tree and tree serialiser components
% visualiser does not need so much detail into the individual components and
% stages

\paragraph{Graph creator}
% {{{

The graph creator component will produce a connected graph of water molecules,
this will then be used for compression and visualisation. Compression will
occur using the graph, while the visualiser will display the graph to aid in
compression.

% }}}

\paragraph{Spanning tree creator}
% {{{

The spanning tree creator component will be responsible for producing a
spanning tree from the graph of water molecules. This spanning tree will be
used for the predictive spanning tree compression algorithm.

% }}}

\paragraph{Tree serialiser}
% {{{

The tree serialiser component will take the spanning tree and convert it to an
easy to compress format.

% }}}

% }}}

\section{Verification components}
% {{{
\label{sec:verification components}

% }}}

% }}}

