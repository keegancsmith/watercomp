\chapter{Introduction}
% {{{
\label{cha:introduction}

Molecular data is composed of many interconnected atoms and interacting
molecules, this poses a problem for visualisation: how do you effectively
visualise a large number of objects in 3D space? The complex diagrams that arise
from basic visualisation techniques will be hard to analyse and identify
structures due to the large number of objects and extraneous data.

Although there are several different visualisation techniques that have been
developed, how they cope with many objects in 3D space will need to be
evaluated. Another aspect that is often neglected is the temporal aspect, data
changes over time and this should also be reflected in the visualisation.

In order to get some perspective on what some of the existing visualisation
techniques are, this paper will look at some of the molecular modelling and
visualisation techniques, as well as some general visualisation techniques
applicable to molecular data.

Chapter \ref{cha:background} will provide background on relevant existing
visualisation techniques.

Chapter \ref{cha:design} will detail why the visualisation program was designed
as it was.

Chapter \ref{cha:implementation} will provide implementation details of the
visualisation program.

Chapter \ref{cha:experiment} will provide details and the results of the
experiment that was carried out.

Chapter \ref{cha:conclusion} will finally conclude and summarise this report.

% }}}

