\setcounter{page}{1}
\pagenumbering{arabic}

\chapter{Introduction}
% {{{
\label{cha:introduction}

Molecular simulations can and do generates lots of data. A large simulation can
comprise of hundreds of thousands of atoms, with thousands of frames.
Multiplying the two dimensions together yields many gigabytes of data. Due to
the computational costs and time required in performing the simulations, the
simulations are often run on clusters, which means that the generated data needs
to be transferred from the cluster to the researchers. Compressing the molecular
data is thus advantageous as it will make the transferring the data faster, as
well as decreasing the storage costs of the molecular simulation data.

A large proportion of the volume in molecular simulations is water, which has
certain characteristics that can be exploited to achieve good compression. The
water compression project thus targets molecular simulations with large amounts
of water, aiming to achieve high compression ratios by exploiting certain
characteristics of the data.

Due to molecular simulation being composed of a sequence of frames of data,
where a single frame is a snapshot of the positions of all the atoms,
compressing the molecular simulation can be broken up into: compressing a
single frame (intraframe compression), and compressing data across frames
(interframe compression).

The water compression project is thus split into three main parts:

\begin{itemize}
  \item Intraframe compression
  \item Interframe compression
  \item Visualisation
\end{itemize}

The visualisation aspect of the project supports the main goal of the project:
compressing molecular simulations. To do so, the molecular simulation data will
need to be visualised, and a quantisation experiment carried out. Quantisation
is converting the original floating point values to integer values, this is to
facilitate compressing the molecular data as integers can be more effectively
compressed.

This report will focus on the visualisation aspect of the water compression
project. The intraframe compression is completed by Keegan Smith, while the
interframe compression is completed by Julian Kenwood.

Chapter \ref{cha:background} provides background on relevant existing
visualisation techniques that influenced the development of the visualisation
aspect of the system.

A number of visualisation approaches has been identified, and is detailed in
Chapter \ref{cha:design}. The chapter details why the approaches are chosen and
the design behind them. A few of the visualisation approaches is aimed at
supporting the compressors.

Chapter \ref{cha:implementation} provides implementation details on each of the
visualisation components.

The second part of the visualisation aspect of the system, the quantisation
experiment, is detailed in Chapter \ref{cha:experiment}. The experiment results
and analysis is also in Chapter \ref{cha:experiment}.

Finally, Chapter \ref{cha:conclusion} concludes and summarises this project.

% }}}

