\chapter{Introduction}
% {{{
\label{cha:introduction}

% Molecular data is composed of many interconnected atoms and interacting
% molecules, this poses a problem for visualisation: how do you effectively
% visualise a large number of objects in 3D space? The complex diagrams that arise
% from basic visualisation techniques will be hard to analyse and identify
% structures due to the large number of objects and extraneous data.

% Although there are several different visualisation techniques that have been
% developed, how they cope with many objects in 3D space will need to be
% evaluated. Another aspect that is often neglected is the temporal aspect, data
% changes over time and this should also be reflected in the visualisation.

% In order to get some perspective on what some of the existing visualisation
% techniques are, this paper will look at some of the molecular modelling and
% visualisation techniques, as well as some general visualisation techniques
% applicable to molecular data.

Molecular simulations can and do generates lots of data. A large simulation can
comprise of hundreds of thousands of atoms, with thousands of frames.
Multiplying the two dimensions together yields many gigabytes of data. Due to
the computational costs and time required in performing the simulations, the
simulations are often run on clusters, which means that the generated data needs
to be transferred from the cluster to the researchers. Compressing the molecular
data is thus advantageous as it will make the transferring the data faster, as
well as decreasing the storage costs of the molecular simulation data.

A large proportion of the volume in molecular simulations is water, which has
certain characteristics that can be exploited to achieve good compression. The
water compression project thus targets molecular simulations with large amounts
of water, aiming to achieve high compression ratios by exploiting certain
characteristics of the data.

The water compression project is split into three main parts:

\begin{itemize}
  \item Interframe compression
  \item Intraframe compression
  \item Visualisation
\end{itemize}

This report will focus on the visualisation aspect of the water compression
project. The interframe compression is completed by Julian Kenwood, while the
intraframe compression is completed by Keegan Smith.

The visualisation supports the compression aspect of the system by visualising
specific areas of the molecular data.  However, the visualisation also provides
more general visualisation uses, not specific to the compression of the
molecular data.

Chapter \ref{cha:background} provides background on the relevant existing
visualisation techniques that influenced the development of the visualisation.

A number of visualisation approaches has been identified, and is detailed in
Chapter \ref{cha:design}. The chapter details why the approaches are chosen and
the design behind them. A few of the visualisation approaches is aimed at
supporting the compressors.

Chapter \ref{cha:implementation} provides implementation details on each of the
visualisation components.

Returning back to supporting the compression aspect of the system, the
compression technique used is lossless, except for the quantisation step. An
experiment was thus carried out to determine the perceptual impact of different
quantisation levels. Chapter \ref{cha:experiment} provides the details and
results from the quantisation experiment.

Chapter \ref{cha:conclusion} finally concludes and summarises this project.

% }}}

