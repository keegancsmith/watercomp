\graphicspath{{./experiment/}}

\chapter{Quantisation Experiment}
% {{{
\label{cha:experiment}

The position data to be compressed in our water compression system is stored as
floating point data. To facilitate compression, the floating point data is
converted to integer values through quantisation. An appropriate number of
bits, the range of integer values, needs to be decided. Fewer bits will result
in higher compression, but more quantisation error will be incurred; while more
bits will result in lower compression, but the data is stored more accurately.

To determine the appropriate level of quantisation, an experiment was conducted
to test the perceptually visible effects of quantisation. More simply stated:
how noticeable are the effects of quantisation? This will help determine the
appropriate level of quantisation; using the least number of bits, while
maintaining visual similarity to the original data.

\paragraph{The experiment hypothesis is:} what is the level of quantisation where the
perceived differences between the original and quantised data are not
significantly different.

The rest of this chapter details various aspects of the experiment.

\section{Introduction}
% {{{
\label{sec:experiment_introduction}

The participants in the experiment were required to rate the difference between
the original unquantised data, and the quantised data. They used a scale of 1
to 7 to rate the difference, 1 being ``Not different'' and 7 being ``Very
different''. Two different visualisation techniques, and four different
quantisation levels were tested. The experimental procedure is detailed in
Section \ref{sec:experiment_procedure}.

Figures \ref{fig:experiment_ballstick4680} and
\ref{fig:experiment_metaballs4680} show the different quantisation levels for
the ball-and-stick, and metaballs visualisation techniques, respectively. The
original unquantised data is on the left, with the four different quantisation
levels; 10 and 8 bit quantisation at the top, 6 and 4 bit quantisation at the
bottom.

\begin{figure}
  \begin{center}
    \includegraphics[width=120mm]{ballstick4680}
  \end{center}
  \caption{Ball-and-stick visualisation showing the different quantisation
  levels}
  \label{fig:experiment_ballstick4680}
\end{figure}

\begin{figure}
  \begin{center}
    \includegraphics[width=120mm]{metaballs4680}
  \end{center}
  \caption{Metaballs visualisation showing the different quantisation
  levels}
  \label{fig:experiment_metaballs4680}
\end{figure}

% }}}

\section{Venue and equipment}
% {{{
\label{sec:experiment_venue}

The participants took part in the experiment one at a time within a closed
room. They watched the data on a 19-inch CRT monitor, which was connected to a
laptop. The laptop was used to load and display the data. The same laptop was
used for all the experiments: \begin{itemize} \item Processor: Intel Core 2 Duo
Processor T6570 2.1GHz \item Memory: 2GB DDR2 \item GPU: Mobile Intel Graphics
Media Accelerator X4500 HD \end{itemize}

% }}}

\section{Participants}
% {{{
\label{sec:experiment_participants}

All the participants were students from the University of Cape Town. Half of
the participants were Science students, with the remaining half consisting
mostly of Commerce and Humanities students. Nineteen of the participants were
male, the remaining 11 participants were female.

The only requirement for a participant to take part in the experiment was that
they not be visually impaired. This is called ``vision corrected to normal''
and is common in perceptual experiments. All that is needed is that they are
able to watch and compare two different sets of data (molecular
visualisations).

A total of 30 participants took part in the experiment. Most of the experiments
were completed in the first three days (25 of the planned 30), while the
remaining experiments were completed over three more days.

Single session for each participant.

% }}}

\section{Variables}
% {{{
\label{sec:experiment_variables}

The variables of the experiment are the four different quantisation levels, and
the two different visualisation techniques.

The four quantisation levels tested were: 4, 6, 8 and 10 bit quantisation.

10 bit quantisation was chosen as the upper quantisation level because beyond
this setting, the quantisation effects are too minimal to be noticeable as
determined by a pilot experiment. Even at 10 bit quantisation, the differences
are miniscule (see Figure \ref{fig:experiment_ballstick4680}).

The two visualisation techniques were: ball-and-stick, and metaballs.

At the time of the experiment, there were four datasets available, all of which
were used and randomly assigned to the participants.

% }}}

\section{Procedure}
% {{{
\label{sec:experiment_procedure}

The following experimental procedure was consistently applied to each of the
participants that took part in the experiment.

\begin{enumerate}

  \item The subject is given a piece of paper explaining the experiment and
  what they are required to do.

  \item A random dataset is selected for the participant.

  \item The original unquantised data is loaded and the subject is allowed to
  explore and view it.

  \item Thereafter, two different quantised data are loaded for the subject to
  view.

  \item The original data is loaded again to remind the participant what the
  original data looked like.

  \item A final two more quantised data are loaded.

  \item The quantisation levels are chosen in a random order.

  \item After showing each quantised data, the subject is asked to compare it
  with the original data, indicating on a scale of 1 to 7, how noticeable the
  differences are: 1 being ``Not different'' and 7 being ``Very different''.

  \item This procedure is repeated for a total of 2 visualisation techniques,
  with 2 different datasets each.

  \item The order of visualisation techniques shown is the same for all
  participants: first ball-and-stick, then metaballs.

\end{enumerate}

On average, the participants took 25 minutes to complete the experiment. The
duration of the experiment varied due to the different datasets. Not all
datasets were of equal size or length, some datasets required more time to play
from start to finish.

After conducting a pilot experiment, it was decided that the participants
should not be able to rotate and explore the data. Rotating the view makes the
data look very different and the participants might become lost and
disoriented. This could invalidate the results of the experiment as the
participants would not be able to accurately compare the unquantised and
quantised data. This becomes more a test of navigation than viewing and prior
experience would play a part.

% }}}

\section{Summary}
% {{{
\label{sec:experiment_summary}

The experiment used the ball-and-stick, and metaballs visualisation techniques;
with 4, 6, 8 and 10 bit quantisation levels. A total of 30 participants took
part in the experiment, each viewing two different datasets. There were four
datasets used in the experiment, thus, each dataset was viewed by 15
participants.

Over the four datasets, a total of 60 scores per quantisation level were
recorded. The results from the experiment are presented and analysed in Chapter
\ref{cha:results}.

% }}}

% }}}

